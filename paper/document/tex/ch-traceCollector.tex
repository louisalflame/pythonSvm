
\chapter{Trace Collector}\label{ch:traceCollector}

\section{SpecElicitor}

Our LAb 建立了一個工具SpecElicitor

使用者可以在GUI介面下一邊建立APP的trace, 同時標記該畫面/動作的label

使用完後會產生含有 XML, 截圖, label的traces

\section{WebTraceCollector}

this paper 用python 和 seleium為基底建立一個可以自動explore website的trace collector

此工具用DOM tree來做state-base的finite state machine

\subsection{framework}

此工具一邊使用selenium來操縱web browser,
一邊建立state(DOM), edge(click ,inputs) 的automata

此工具使用event list的形式,選擇一個algorithm,每次pop out一個待辦的event
依據event執行click和inputs

要執行click action時,會先對所有的input 填值,會連上database進行input,clickable的tag等feature的字串處理,找出推薦的適當的值填入


\subsection{algorithm}

根據不同的algorithm,此工具面對可按的clickables

\subsubsection{Monkey}

每次update:

\subsubsection{Depth First Search}

每次

\subsection{automata}

一個state

一個edge

一個trace
