
\chapter{Experiments}\label{ch:experiments}

We construct a tool WebTraceCollector for automatic web testing,
which can explore the website automaically and 
analyze the DOM tree of the current web page to find the next clickable element to click,
and propose a machine learning method to evaluate traces by SVM.
In order to use the tool and implement the evaluation, wee need a laptop and installing some tools.
The specification of the laptop we used is shown in Table[\ref{laptopSpec}],
and we need to install Windows, Python 3.4.3 and MySQL database.

\begin{table}[ht]
	\begin{center}
		\begin{tabular}{| l | l | }
			\hline
			Laptop & AUSU X450JN \\ \hline
			Operating System & Windows 10 (64bit) \\ \hline
			CPU & Intel(R) Core(TM) i5-4200 @2.8GHz \\ \hline
			RAM & 4GB \\ \hline
			Storage & 800G \\ \hline
		\end{tabular}
		\caption{ The specification of the laptop. }
		\label{laptopSpec}
	\end{center}
\end{table}

WebTraceCollector is constructed based on Python,
it control the browser by the selenium library and recognize the elements of the current page by the beautifulsoup library.
The automatic exploring mechanism highly depends on the page source downloaded from the current web page.
If the page source ot the constructed DOM tree can not match the current page correctly,
we can not find the correct clickable element and the testing may fail.
Thus, it is important to check how many web applications can download page soucre and construct DOM tree accurately.
In the experiment 1, we find 20 websites which is famous in taiwan.
We test those websites and generate traces automatically by WebTraceCollector.
Moreover, we want to check the ability of handling dynamic web pages.
We choose some common web pages with lots of input fields need to be inserted to test in the experiment 2.
In the experiment 3, we want to check the accuracy of the automatic prediction.
We generate some traces of certain applications and use those traces to train SVM model,
and we use this model to predict the traces of the applications.

\section{Trace collection}

In 20 popular applications\cite{popularWebs} that shown in Table \ref{popularWebs},
there are forums, news, community platforms, games and blogs.
We generate monkey traces by randomly click buttons and the length of traces is about 5.
The reason that we do not completely explore the whole website is most of the scale of websites are too big.
For example, a news website may have thousands of subpages and it may cost too much time to generate completely traces.
We just want to check the DOM tree is constructed correctly, 
so we make short traces to focus on checking the ability of clicking correct clickable elements.

The result of the experiments is shown in Table[].
There are 4 web applications that have serious problem on testing,
and most of web applications can be successfully explored with little defects.
There are two unkown fail happened, however it can work after restart the test.
The problems of the fail applications are 
too many iframes of advertising and API plugin in the webpages and buttons functions implemented by javascript.
The iframes become noise in the page source and make the DOM tree constructed incorrectly,
so WebTraceCollector may not find the correct clickable elements or only focus on the clickables in the advertising iframe.
On the other hand, the button functions in the web pages may implemented by javascript.
That means clickable elements may not be the style we thought, such as link tags or input-button tags,
it could be images, div tags or any tags if javascript bind the mouse listener on.
With the uncertain clickable elements, it is hard to find the correct clickable elements to explore the web and record the traces.
WebTraceCollector will click the wrong elements with nothing happens and stay at the same page until the trace is end.

\begin{table}[ht]
	\begin{center}
		\begin{tabular}{ | l | l | l | l | }
			\hline
			編號 & 網站名稱 & 網站類型 & URL \\ \hline
			1 & Facebook & Community platform & https://www.facebook.com/ \\ \hline
			2 & YouTube & Entertainment & https://www.youtube.com/ \\ \hline
			3 & Yahoo & Search engine & https://tw.yahoo.com/ \\ \hline
			4 & Google & Search engine & https://www.google.com.tw/ \\ \hline
			5 & 中時電子報 & News & http://www.chinatimes.com/ \\ \hline
			6 & 露天拍賣 & Commerce & http://www.ruten.com.tw/ \\ \hline
			7 & 聯合新聞網 & News & http://udn.com/news/index \\ \hline
			8 & 巴哈姆特 & Forum & http://www.gamer.com.tw/ \\ \hline
			9 & Mobile01 & Forum & http://www.mobile01.com/ \\ \hline
			10 & 蘋果日報 & News & http://www.appledaily.com.tw/ \\ \hline
			11 & 百度 & Search engine & https://www.baidu.com/ \\ \hline
			12 & 東森新聞雲 & News & http://www.ettoday.net/ \\ \hline
			13 & 卡提諾論壇 & Forum & http://ck101.com/ \\ \hline
			14 & 伊莉討論區 & Forum & http://www40.eyny.com/index.php \\ \hline
			15 & Hinet & Search engine & http://www.hinet.net/ \\ \hline
			16 & 微軟Live.com & Search engine & https://login.live.com/ \\ \hline
			17 & 痞客邦 & Blog & https://www.pixnet.net/ \\ \hline
			18 & PChome Online & Search engine & http://pchome.com.tw/ \\ \hline
			19 & 淘寶 & Commerce & https://world.taobao.com/ \\ \hline
			20 & Life生活網 & Blog & http://www.life.com.tw/ \\ \hline
		\end{tabular}
		\caption{ The 20 popular websites. }
		\label{popularWebs}
	\end{center}
\end{table}

%針對20個網站做20個monkey traces的結果
%有些網站太複雜 iframe太多,太相似

\clearpage

\section{Dynamic webpages}

In order to test the ability of explore dynamic webpages,
we select some webpages that have lots of input fields and it need to insert correct values to pass through the web pages.
The selected webpages are shown in Table \ref{DynamicWebs}.
Most of the input fields in those web pages are names, years, emails and so on.
We can see that there are 4 web pages can be successfully passed through, but 3 web pages can not.
There are 3 reasons that can not pass through the dynamic web pages are the follows:
\\	1. The web pages have robot checking mechanism to prevent automatic exploring.
\\	2. The form of input fields are different to the examples in the database.
\\	3. The input fields are out of the range of examples we collecteed.
	
In reason 1, the robot checking like image recognition is developed to prevent people exploring the website automatically be computer.
It is too hard to find out the correct value by program,
and it is unreasonable if we can easily pass through those web pages.
If we want to test those websites,
the reasonable way is let the developers of the websites turn off the robot ckecking.

In reason 2, even though we have collect the examples of the input fields in the database, 
it is still a hard work to find out the correct form of the values.
For instance, we analyze the input field and find out the value should be a string of address.
But the type of input fields can be text, checkbox or selects, 
it is too hard to analyze user should insert the whole string of address or select each partens of area.

In reason 3, the web pages can have unlimited kinds of input fields.
Although we can add as more examples and rules in the database as we possible,
it is still impossible to handle all kinds of input fields by only few kinds of examples.
In future work, we can develop to find out the values by other method such as maching learning.


\begin{table}[ht]
	\begin{center}
		\begin{tabular}{ | l | l | }
			\hline
			URL & Result \\ \hline
			http://www.plurk.com/signup & Pass \\  \hline
			https://ups.moe.edu.tw/Personal\_Page/index.php & Pass \\  \hline
			https://applyweb.collegenet.com/account/new/create  & Pass \\  \hline
			https://apply.grad.ucsd.edu/signup  & Pass \\  \hline
			https://user.gamer.com.tw/regS1.phpt  & Fail \\  \hline
			http://panel.pixnet.cc/signup/step2  & Fail \\  \hline   
			https://apps.grad.uw.edu/applForAdmiss/newUserProfile.aspx?bhcp=1 & Fail \\  \hline 
		\end{tabular}
		\caption{ The dynamic webpages. }
		\label{DynamicWebs}
	\end{center}
\end{table}
%針對5個有inputs的網站做traces
%有的網站可以通過 有的只能一次(註冊) 有的偶爾可以(登入) 有的不行


%\section{Prediction}

%對同種類的網站 APP 和WEB 做traces
%測出其中有BUG

