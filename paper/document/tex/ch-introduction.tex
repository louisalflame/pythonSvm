
\chapter{Introduction}\label{ch:introduction}

\section{Background}

Web Applications become more and more indispensable in our life.
People can easily get informations with the development of the Internet.
It becomes more convenient for people to search new knowledge, buy goods and chat with friends.
People spent lots of time on the Internet for working, studying, shopping, socialozong and playing.
As the result, more engineers tend to develop the applications on the Internet.
On the other hand, Mobile Applications is also a choice for software developer.
People can carry smart phone and enjoy the applications everywhere.
Although people can visit web applications by browser on the smart phone,
a special mobile application has a better performance.
Companies may develop the applications on both the Internet and the mobiles.
In order to guarantee the performance and user experience of the applications,
the testing of the applications is required.
With generating lot of test cases, 
it can help engineers find the bugs in their applications from the failed test cases.

\clearpage

\section{Motivation}

There are serveral tools developed fro testing applications nowadays,
for example, developers can use selenium to test that their website is working successfully.
However, the current techniques of testing take lots of works and times.
Because the applications are complicated and dynamic,
they have different reaction with different inputs.
Engineers need to write the test scripts case by case according to the each functions.
They also need basic knowledge of software testing.
To reduce the work time of testing, the automated testing become essential for engineers.
But the challenge is that it is too difficult to generate test case on black box testing.
Without the detailed information of functions,
We may not know how to make a test script.
Thus, to find a method for solving the problem of automated tesing dynamic applications is important.
The automtated testing should help developers generate a variety of traces 
and try to predict which traces is failed and 


%介紹software testing,其中又有web testing和mobile APP testing
%software變化大,需要測試確保功能正常,但製作testcase耗費時間人力
%auto testing的好處: 節省人力 black-box testing的好處:應用到所有APP
%建立一個auto testing ,開發者可減輕開發成本


\section{Purpose}

In this paper, we propose a technique to automatically generating traces on dynamic web applications.
The program named WebTraceCollector can collect the informations on web pages and try to guess the suitable inputs.
We construct a input databank with some examples of input,
so the program will find a similar example as the fitting input value of the web page.
With the suggested input values,
the program can explore the website automatically and build the finite state machine to represent the website,
and record the every step and web pages during the testing and generate the traces for testing.

In order to automatically evaluate traces collected from web and mobile applications,
we propose a teachnique to extract the feature from the traces and transform the traces into feature vector.
We constrcut a keyword dictionary, 
which collects keywords that represent the common sense behaviors of action and screen from applications,
so the traces can be expressed in a vector format.
Then, we use the method of machine learning to evaluate traces.
We use SVM to classify the traces into passed traces and failed traces.


%本篇建立了一個auto web testing的工具,和測試web,APP並驗證的framework
%framework將testcase取出symbolic label,train出所有APP可通用的model來evaluate all traces 


\clearpage

\section{Organization}

The rest of this paper is organized as follows.
Chapter 2 shows the related work of testing, 
and we introduce the tool for generating mobile traces and the method used for evaluating in Chapter 3.
In Chapter 4, we propose a technique to generate traces on dynamic web applications.
Chapter 5 shows the framework to evaluate traces from the web and mobiles.
Then, the experiment results of collecting traces from web applications and 
predicting traces are shown on the Chapte 6,
and the conclusion is shown in Chapter 7.

%第二章 related work: 其他web,APP tesing, ex: selenium,crwaljax,splinter,Jmeter,UIautomata...
%第三章 preliminaries: machine learning,, test evaluation
%第四章
%第五章

